\documentclass[fancyhdr,bookmark]{ctexbook}
\setCJKmainfont{SimSun}
\setmainfont{Georgia} 	% 設定英文字型
\setromanfont{Georgia} 	% 字型
\setmonofont{Latin Modern Mono}

% pandoc版本大于1.15时需要\tightlist
\providecommand{\tightlist}{%
  \setlength{\itemsep}{0pt}\setlength{\parskip}{0pt}}

\usepackage{tikz} % Required for drawing custom shapes
\usepackage[yyyymmdd,hhmmss]{datetime}
\ctexset{today=small}
\usepackage{geometry} 		% 設定邊界
\geometry{
  top=1in,
  inner=1in,
  outer=1in,
  bottom=1in,
  headheight=3ex,
  headsep=2ex
}


\usepackage{xcolor}
\definecolor{ocre}{RGB}{243,102,25} % Define the orange color used for highlighting throughout the book

%\usepackage[x11names,svgnames,dvipsnames]{xcolor}
\usepackage{listings}
\lstset{
	%numbers=left,
	%numberstyle=\tiny,
	basicstyle=\small\ttfamily,
	keywordstyle=\color[rgb]{0.13,0.29,0.53}\textbf,
	commentstyle=\color[rgb]{0.56,0.35,0.01}\textit,
	identifierstyle=\color[rgb]{0.00,0.00,0.00},
	stringstyle=\color[rgb]{0.31,0.60,0.02},
	frame=shadowbox,
	rulesepcolor=\color{red!20!green!20!blue!20},
	backgroundcolor=\color[rgb]{0.97,0.97,0.97},
	tabsize=4,
	breaklines=tr,
	showstringspaces=false,
}
\renewcommand{\lstlistingname}{代码}

%\newcommand{\KeywordTok}[1]{\textcolor[rgb]{0.13,0.29,0.53}{\textbf{{#1}}}}
%\newcommand{\DataTypeTok}[1]{\textcolor[rgb]{0.13,0.29,0.53}{{#1}}}
%\newcommand{\DecValTok}[1]{\textcolor[rgb]{0.00,0.00,0.81}{{#1}}}
%\newcommand{\BaseNTok}[1]{\textcolor[rgb]{0.00,0.00,0.81}{{#1}}}
%\newcommand{\FloatTok}[1]{\textcolor[rgb]{0.00,0.00,0.81}{{#1}}}
%\newcommand{\CharTok}[1]{\textcolor[rgb]{0.31,0.60,0.02}{{#1}}}
%\newcommand{\StringTok}[1]{\textcolor[rgb]{0.31,0.60,0.02}{{#1}}}
%\newcommand{\CommentTok}[1]{\textcolor[rgb]{0.56,0.35,0.01}{\textit{{#1}}}}
%\newcommand{\OtherTok}[1]{\textcolor[rgb]{0.56,0.35,0.01}{{#1}}}
%\newcommand{\AlertTok}[1]{\textcolor[rgb]{0.94,0.16,0.16}{{#1}}}
%\newcommand{\FunctionTok}[1]{\textcolor[rgb]{0.00,0.00,0.00}{{#1}}}
%\newcommand{\RegionMarkerTok}[1]{{#1}}
%\newcommand{\ErrorTok}[1]{\textbf{{#1}}}
%\newcommand{\NormalTok}[1]{{#1}}



\ifxetex
  \usepackage[setpagesize=false, % page size defined by xetex
              unicode=false, % unicode breaks when used with xetex
              xetex]{hyperref}
\else
  \usepackage[unicode=true]{hyperref}
\fi
\hypersetup{breaklinks=true,
            bookmarks=true,
            pdfauthor={An He},
            pdftitle={iTop实施杂记},
            colorlinks=true,
            urlcolor=blue,
            linkcolor=magenta,
            pdfborder={0 0 0}}
\urlstyle{same}  % don't use monospace font for urls
% Make links footnotes instead of hotlinks:
\renewcommand{\href}[2]{#2\footnote{\url{#1}}}



\title{iTop实施杂记}
\author{An He}
\date{\today}

\usepackage{fancyhdr}
%\usepackage{lastpage}
\pagestyle{fancy}


\begin{document}
\frontmatter
%----------------------------------------------------------------------------------------
%	TITLE PAGE
%----------------------------------------------------------------------------------------

\begingroup
\thispagestyle{empty}
\begin{tikzpicture}[remember picture,overlay]
\node[inner sep=0pt] (background) at (current page.center) {\includegraphics[width=\paperwidth]{Pictures/background}};
\draw (current page.center) node [fill=ocre!30!white,fill opacity=0.6,text opacity=1,inner sep=1cm]
{\Huge\centering\bfseries\sffamily\parbox[c][][t]{\paperwidth}
{\centering iTop实施杂记\\[13pt] % Book title
{\huge An He} % Author name
}};
\end{tikzpicture}
\vfill
\endgroup
\addcontentsline{toc}{chapter}{封面}

%----------------------------------------------------------------------------------------
%	COPYRIGHT PAGE
%----------------------------------------------------------------------------------------
\newpage
~\vfill
\thispagestyle{empty}

\noindent Copyright \copyright\ \the\year\  An He\\ % Copyright notice

\noindent \textsc{Published by \LaTeX}\\ % Publisher
\noindent \textsc{https://github.com/annProg/itop-book}\\ % URL

\noindent Licensed under the Creative Commons Attribution-NonCommercial 3.0 Unported License (the ``License''). You may not use this file except in compliance with the License. You may obtain a copy of the License at \url{http://creativecommons.org/licenses/by-nc/3.0}. Unless required by applicable law or agreed to in writing, software distributed under the License is distributed on an \textsc{``as is'' basis, without warranties or conditions of any kind}, either express or implied. See the License for the specific language governing permissions and limitations under the License.\\ % License information

\noindent \textit{最后编译日期, \today\ \currenttime } % Printing/edition date


    
\chapter*{前言}
\addcontentsline{toc}{chapter}{前言}
从事运维工作已经3年多,所见所闻,切身感受,当运维的规模大到一定程度的时候,
资源的管控将变得很重要,主要体现在效率和成本两个方面。

举例来说,我接手了一个URL监控的工作,具体流程是:研发人员写一个监控配置文件提交到
代码库,然后发邮件通知我,我执行一个脚本将监控加入Zabbix。当规模很小的时候不会有
什么问题,即使完全手动来配置都可以,我也能记住
所有的URL。一旦规模大到一定程度,比如说50个,我记不住了,或者每周每月都有几十的增减,
需要经常投入时间去执行脚本,研发人员配置文件写错的时候还要来回邮件沟通。这个
时候有一个合理的管控手段就非常重要了。要有一个系统来记录URL的监控信息以及状态,同步
给监控系统,并且URL的增删改应该由研发自助完成,无需运维人肉参与,这样一来,监控变
更速度(效率)提升了,运维也不用做这个重复的工作了。人力成本,沟通成本相应就下降了。

一个反例,我第二份工作,公司层面没有一个有效全面的资源管控的技术手段,没有全
公司统一的业务线口径。服务器、
数据库、域名、缓存等资源一般只能管理到部门粒度,就是说一个部门指定几个接口人,
有事都找他们。这样的设计是很糟糕的,资源提供方把本该自己牢牢掌控的资源
管理权下放给了各个部门,导致自己无法掌握资源的关联及使用情况,业务部门受限于人力
或者技术,也常常不能有效的管控资源,常常是业务早下线了,数据库还长期占用着资源。
在公司经历
危机之后,拖欠了好几个机房的费用,这时开始大张旗鼓手忙脚乱的回收闲置资源或者关闭机房,
统计资源的手段用的却是Excel表格、编辑wiki页面等形式。这种混乱带来的不仅仅是效率
低下,还有高昂的运维成本以及负反馈带来的恶性循环:资源无法得到有效的管控,不能及时回收,
需要不断的购入新的服务器资源,最终成本只会越来越高。

我认为正确的方向
应该是用技术手段来减小资源管理的粒度,统一业务线名称,并用一个系统去搞清楚
资源关联关系,而不是Excel
表格。具体来说,业务要挂在具体的人名下,而资源要挂在业务下面,而不是某个人名下,
这样就能容易的做到业务下线,相关资源一并下线。还带来一个额外的好处,人员离职只
需要交接业务,因为资源关联的是业务而不是人,就不会出现人员离职找不到联系人的问题。
这样不仅仅是减少了钱的支出,也提高了沟通的效率。
只要粒度够细,关联关系够清晰,很多事情都会迎刃而解,比如报警,有这样一个CI关联
数据库就不愁报警发给谁了,就不用去人肉维护Zabbix动作以及用户了。

针对以上问题,我希望引入CMDB系统来解决。调研过几个开源的CMDB系统,包括yourcmdb,i-doit等。
最终选择了文档功能都较为
完善,开发也较为活跃的iTop作为CMDB工具进行定制,管理资源的关联关系,
实现业务联系人查询接口供报警调用,避免用
zabbix维护报警联系人;定制iTop权限让研发人员可以编辑自己负责的APP,相当于自助订阅报警;
将iTop作为URL监控配置的前端表单,实现研发自助添加URL监控;
定制工单系统,自动指派工单,自动录入新申请资源。
实际效果是我部门的资源管理情况得到改善。由于没有机会在整个公司层面实施,有些观点可能
并不全面。另外,我对ITIL理论并不熟悉,因此不打算从理论
上探讨CMDB,只从iTop的实施经历,谈谈iTop
CMDB的定制与运维自动化的一些经验。

计划安排以下内容:iTop简介,iTop插件开发流程,常用插件介绍;然后通过几个案例说明iTop对
运维效率提升机成本优化的帮助,包含一致性维护,工单流程,标准化,CI关联关系,公共查询接口等主题。

\textbf{第一章} 简单介绍一下iTop,iTop的插件开发流程,CI定制,本地化等等
\textbf{第二章} CI属性约束,唯一性、profile权限,只读,隐藏等等
\textbf{第三章} SSO集成方法 \textbf{第四章} iTop简介 iTop插件开发流程
CI定制(删除、新增、修改) menunode定制 本地化
CI属性约束(唯一性,只读,隐藏) SSO集成方法 action-shell-exec trigger
CI生命周期(lifecycle) cmdbapi rest扩展 api-client
request-template,工单自动指派,资源自动入库 custom-pages iframe嵌入
AttributeClassCustom 其他常用插件介绍 审计 Profile,自助化与权限控制
实例(URL监控) api相关的可以放在系统集成部分


{
\hypersetup{linkcolor=black}
\setcounter{tocdepth}{2}
\tableofcontents
\addcontentsline{toc}{chapter}{目录}
}
\listoftables
\addcontentsline{toc}{chapter}{表格列表}
\listoffigures
\addcontentsline{toc}{chapter}{插图列表}



\mainmatter
    % 在此命令之后的页码为阿拉伯数字
    % 以下为正文
\chapter{SSO集成方法}\label{ssoux96c6ux6210ux65b9ux6cd5}

\chapter{CI校验}\label{ciux6821ux9a8c}

\chapter{action-shell-exec}\label{action-shell-exec}

\backmatter

\end{document}
